%==== 文档类 ====================================================================
\documentclass[openright,twocolumn]{book} % openright: 章从右页开始; twocolumn: 全文双栏

%==== 数学 & 定理宏包 ==========================================================
\usepackage{amsmath}    % 数学环境
\usepackage{amssymb}    % 额外数学符号
\usepackage{amsfonts}   % 额外数学字体
\usepackage{amsbsy}     % 粗体数学符号
\usepackage{amsthm}     % 基本定理环境
\usepackage{thmtools}   % 高级定理样式
\usepackage{mathrsfs}   % \mathscr 花体
\usepackage{tensor}     % 张量上下标
\usepackage{relsize}    % 相对字号(\mathsmaller 等)
\usepackage{bm}

%==== 字体 & 引擎选择 + 中文支持 ===============================================
\usepackage{iftex}
\ifTUTeX
  %=== XeLaTeX / LuaLaTeX 一类编译:强烈推荐用于中英混排 ========================
  \usepackage{unicode-math}
  \usepackage{xeCJK}                       % 中文支持(自动处理 UTF-8 中文)
  \setmainfont{TeX Gyre Termes}[Scale=1.0] % 英文字体
  \setmathfont{TeX Gyre Termes Math}       % 数学字体
  % 把下面的中文字体名改成你系统中存在的字体:
  % Windows 常见: SimSun / SimHei / Microsoft YaHei
  % macOS  常见: Songti SC / STSong / Heiti SC
  \setCJKmainfont{SimSun}                  % 中文正文字体
  \setCJKfamilyfont{hei}{SimHei}           % 中文黑体
  \newcommand{\hei}{\CJKfamily{hei}}       % 切换到黑体:\hei 文字
\else
  %=== pdfLaTeX 编译:通过 CJK 宏包支持中文 =====================================
  \usepackage{txfonts}                     % 老的 Times 字体(可按需删)
  \usepackage{newtxtext,newtxmath}         % Times 风格英文字体 + 数学
  \usepackage{CJKutf8}                     % 中文支持(需 UTF-8 源文件)

  % 自动用 CJK 环境包住整篇文档,无需手动写 \begin{CJK}...\end{CJK}
  \AtBeginDocument{\begin{CJK}{UTF8}{gbsn}}  % gbsn: 简体中文宋体编码
  \AtEndDocument{\end{CJK}}
\fi

\usepackage{anyfontsize}        % 任意字号

%==== 图形 & 绘图 ==============================================================
\usepackage{graphicx}           % 插图
\usepackage{tikz}               % 绘图
\usepackage{pict2e}             % 改进 picture 环境(自定义大号运算符用到)

%==== 颜色设置(宏包 + 自定义颜色)============================================
\usepackage{color}              % 基本颜色支持
\definecolor{dkgreen}{rgb}{0,0.6,0}
\definecolor{gray}{rgb}{0.5,0.5,0.5}
\definecolor{mauve}{rgb}{0.58,0,0.82}
\definecolor{pk}{RGB}{242, 51, 175}
\definecolor{or}{RGB}{140, 136, 28}
\definecolor{mbg}{RGB}{222,237,250}
\definecolor{gd}{RGB}{237,130,0}

%==== 语言、编码、引号(中英混排友好)=========================================
\usepackage[utf8]{inputenc}     % 源文件 UTF-8 编码(pdfLaTeX 需要;Xe/Lua 可忽略)
\usepackage[english]{babel}     % 英文断词;中文由 xeCJK 或 CJK 处理
\usepackage{csquotes}           % 引号/引用样式

%==== 标题、目录、页眉脚、列表、表格等 =======================================
\usepackage{titlesec}           % 标题格式
\usepackage{titletoc}           % 目录格式
\usepackage{fancyhdr}           % 页眉页脚(暂未使用,可按需启用)
\usepackage{enumitem}           % 列表格式
\usepackage{array}              % 表格列类型
\usepackage{epigraph}
\usepackage[UTF8]{ctex}
\newcommand{\xingkai}{\CJKfamily{xingkai}}

%==== 行距、脚注、文本辅助 =====================================================
\usepackage{setspace}           % 行距
\usepackage[perpage]{footmisc}  % 每页脚注重置计数
\usepackage{lipsum}             % 虚拟文本测试排版

%==== 代码高亮(listings)=====================================================
\usepackage{listings}
\lstset{
  frame=tb,
  language=Matlab,              % 默认语言(可在局部改)
  aboveskip=3mm,
  belowskip=3mm,
  showstringspaces=false,
  columns=flexible,
  basicstyle={\small\ttfamily},
  numbers=none,
  numberstyle=\color{gray},
  keywordstyle=\color{blue},
  commentstyle=\color{dkgreen},
  stringstyle=\color{mauve},
  breaklines=true,
  breakatwhitespace=true,
  tabsize=3
}
\lstdefinestyle{output}{
  basicstyle=\ttfamily\small,
  columns=fullflexible,
  keepspaces=true,
  showstringspaces=false,
  breaklines=true,
  frame=single,
  rulecolor=\color{black!20},
  backgroundcolor=\color{black!3},
  frameround=tttt,
  xleftmargin=0.6em,
  xrightmargin=0.6em,
  aboveskip=0.8em,
  belowskip=0.8em,
  lineskip=0.1em,
}
\lstnewenvironment{pyout}
  {\lstset{style=output}}
  {}

%==== 按章节拆分文件:subfiles ================================================
\usepackage{subfiles}
% 主文件 main.tex 中示例:
%   \chapter{Title}
%   \subfile{chapters/ch1.tex}
% 子文件 chapters/ch1.tex 中:
%   \documentclass[../main.tex]{subfiles}
%   \begin{document}
%   ... 内容 ...
%   \end{document}

%==== 自定义大号运算符 + 公式左右对齐控制 ====================================
\newif\ifnarrow
\narrowfalse

\makeatletter
% 通用“放大运算符”构造
\newcommand{\my@big}[1]{%
  \mathop{\vphantom{\sum}\mathpalette\my@makebig{#1}}\slimits@%
}
\AtBeginDocument{%
  \DeclareRobustCommand{\bigplus}{\narrowfalse\DOTSB\my@big\my@plus}%
  \DeclareRobustCommand{\bigtimes}{\narrowtrue\DOTSB\my@big\my@times}%
  \DeclareRobustCommand{\bigbox}{\narrowfalse\DOTSB\my@big\my@box}%
}
\newcommand{\my@makebig}[2]{%
  \ifnarrow
    \def\scalefactor{0.8}%
  \else
    \def\scalefactor{1}%
  \fi%
  \vcenter{%
    \sbox\z@{$\m@th#1\sum$}%
    \setlength{\unitlength}{0.9\dimexpr\ht\z@+\dp\z@}%
    \hbox{\kern0.1\wd\z@\scalebox{\scalefactor}[1]{\my@draw{#1}{#2}}\kern0.1\wd\z@}%
  }%
}
\newcommand{\my@draw}[2]{%
  \begin{picture}(1,1)
    \linethickness{%
      \ifx#1\displaystyle 1.15\fontdimen8\textfont3\else
      \ifx#1\textstyle 1.05\fontdimen8\textfont3\else
      \ifx#1\scriptstyle1\fontdimen8\scriptfont3\else
      1\fontdimen8\scriptscriptfont3\fi\fi\fi
    }%
    #2
  \end{picture}%
}
\newcommand{\my@plus}{%
  \roundcap
  \Line(0.5,0)(0.5,1)
  \Line(0,0.5)(1,0.5)
}
\newcommand{\my@times}{%
  \roundcap
  \Line(0,0)(1,1)
  \Line(0,1)(1,0)
}
\newcommand{\my@box}{%
  \roundcap
  \Line(0,0)(1,0)
  \Line(1,0)(1,1)
  \Line(1,1)(0,1)
  \Line(0,1)(0,0)
}

% 公式左对齐 / 居中切换
\newcommand{\mathleft}{\@fleqntrue\@mathmargin0pt}
\newcommand{\mathcenter}{\@fleqnfalse}
\makeatother

%==== 定理环境样式 ==============================================================

% Theorem
\declaretheoremstyle[
 parent=section,
 headfont=\bf\itshape\normalsize\color{blue},
 headpunct={},
 notefont=\bf\itshape\color{gd}, notebraces={}{} ,
 bodyfont=\small\normalfont
    \setlength{\abovedisplayskip}{3pt}
    \setlength{\belowdisplayskip}{3pt}
    \setlength{\abovedisplayshortskip}{3pt}
    \setlength{\belowdisplayshortskip}{3pt},
 name=Theorem,
 postheadspace=\newline,
 headformat=\NUMBER~\NAME:\NOTE
]{temp1}
\declaretheorem[style=temp1]{theo}

% Definition(与 theo 共用计数)
\declaretheoremstyle[
 sibling=theo,
 headfont=\bf\itshape\normalsize\color{red},
 headpunct={},
 notefont=\bf\itshape\color{gd}, notebraces={}{} ,
 bodyfont=\small\normalfont
    \setlength{\abovedisplayskip}{3pt}
    \setlength{\belowdisplayskip}{3pt}
    \setlength{\abovedisplayshortskip}{3pt}
    \setlength{\belowdisplayshortskip}{3pt},
 name=Definition,
 postheadspace=\newline,
 headformat=\NUMBER~\NAME:\NOTE
]{temp2}
\declaretheorem[style=temp2]{de}

% Remark(无编号)
\declaretheoremstyle[
 numbered=no,
 headformat=\NAME:\NOTE,
 notefont=\bf\itshape\color{gd}, notebraces={(}{)} ,
 headfont=\bf\itshape\normalsize\color{pk},
 headpunct={},
 bodyfont=\small\normalfont
    \setlength{\abovedisplayskip}{3pt}
    \setlength{\belowdisplayskip}{3pt}
    \setlength{\abovedisplayshortskip}{3pt}
    \setlength{\belowdisplayshortskip}{3pt},
 name=Remark,
 postheadspace=\newline
]{temp3}
\declaretheorem[style=temp3]{re}

% Proposition(与 theo 共用计数)
\declaretheoremstyle[
 sibling=theo,
 headfont=\bf\itshape\normalsize\color{blue},
 headpunct={},
 notefont=\bf\itshape\color{gd}, notebraces={}{} ,
 bodyfont=\small\normalfont
    \setlength{\abovedisplayskip}{3pt}
    \setlength{\belowdisplayskip}{3pt}
    \setlength{\abovedisplayshortskip}{3pt}
    \setlength{\belowdisplayshortskip}{3pt},
 name=Proposition,
 postheadspace=\newline,
 headformat=\NUMBER~\NAME:\NOTE
]{temp4}
\declaretheorem[style=temp4]{pro}

% Corollary(以 theo 为 parent)
\declaretheoremstyle[
 parent=theo,
 headfont=\bf\itshape\normalsize\color{mauve},
 headpunct={},
 notefont=\bf\itshape\color{gd}, notebraces={}{} ,
 bodyfont=\small\normalfont
    \setlength{\abovedisplayskip}{3pt}
    \setlength{\belowdisplayskip}{3pt}
    \setlength{\abovedisplayshortskip}{3pt}
    \setlength{\belowdisplayshortskip}{3pt},
 name=Corollary,
 postheadspace=\newline,
 headformat=\NUMBER~\NAME:\NOTE
]{temp5}
\declaretheorem[style=temp5]{co}

% Lemma(与 theo 共用计数)
\declaretheoremstyle[
 sibling=theo,
 headfont=\bf\itshape\normalsize\color{mauve},
 headpunct={},
 notefont=\bf\itshape\color{gd}, notebraces={}{} ,
 bodyfont=\small\normalfont
    \setlength{\abovedisplayskip}{3pt}
    \setlength{\belowdisplayskip}{3pt}
    \setlength{\abovedisplayshortskip}{3pt}
    \setlength{\belowdisplayshortskip}{3pt},
 name=Lemma,
 postheadspace=\newline,
 headformat=\NUMBER~\NAME:\NOTE
]{temp6}
\declaretheorem[style=temp6]{lem}

% Proof(带彩色“Proof”和方块结尾)
\declaretheoremstyle[
 numbered=no,
 headfont=\bf\itshape\color{dkgreen},
 headpunct={},
 bodyfont=\small\normalfont
    \setlength{\abovedisplayskip}{3pt}
    \setlength{\belowdisplayskip}{3pt}
    \setlength{\abovedisplayshortskip}{3pt}
    \setlength{\belowdisplayshortskip}{3pt},
 name=Proof,
 headformat=\NAME:\NOTE,
 postheadspace=\newline,
 qed=$\square$
]{temp7}
\declaretheorem[style=temp7]{p}

%==== 版面尺寸 & 双栏间距 ======================================================
\setlength{\columnsep}{10mm}  % 双栏间距

\hoffset=-20.4mm
\voffset=-25.4mm
\footskip=1truecm
\textwidth=204.4mm
\textheight=24.5truecm
\oddsidemargin=0mm
\evensidemargin=0mm
\marginparsep=0mm
\marginparwidth=0mm
\marginparpush=0mm

\parindent=0pt              % 段首不缩进

%==== 章节编号 & 目录样式 ======================================================
\renewcommand\thesection{\S\arabic{section}}          % §1, §2, ...
\renewcommand\thesubsection{\color{red}$\dagger$\alph{subsection}} % †a, †b, ...
\renewcommand\thepart{}                             % part 不显示编号

\titlecontents{part}[0em]{\bigskip\centering\bfseries}
  {}{}{\hfill}

\titlecontents{chapter}[1.05em]{\bigskip}%
  {\contentslabel[\MakeUppercase{\romannumeral\thecontentslabel}]{2em}\enspace\textsc}%
  {\hspace*{-1em}\textsc}%
  {}%

\titlecontents{section}[1.6em]{\smallskip}%
  {\thecontentslabel.\enspace}
  {}%
  {}%

%==== 常用数学命令 & 运算符 =====================================================
\newcommand*{\defeq}{\stackrel{\mathsmaller{\mathsf{def}}}{{=}}}
\newcommand*{\ideto}{\stackrel{\mathsmaller{\mathsf{ide}}}{{\to}}}
\newcommand*{\cardeq}{\cong_{\mathsmaller{set.}}}
\newcommand*{\topeq}{\cong_{\mathsmaller{top.}}}
\newcommand*{\oo}{\text{\o}}

% 函数限制/共限制
\newcommand\restr[2]{{%
  \left.\kern-\nulldelimiterspace
  #1
  \littletaller
  \right|_{#2}
}}
\newcommand\pd[2]{{%
  \frac{\partial #2}{\partial #1}
}}
\newcommand\cores[2]{{%
  \left.\kern-\nulldelimiterspace
  #1
  \littletaller
  \right|^{#2}
}}
\newcommand\tores[3]{{%
  \left.\kern-\nulldelimiterspace
  #1
  \littletaller
  \right|_{#2}^{#3}
}}
\newcommand{\littletaller}{\mathchoice{\vphantom{\big|}}{}{}{}}

% 括号、花体、黑板体
\newcommand\s[1]{\left\{#1\right\}}
\newcommand\f[1]{\left(#1\right)}
\newcommand\e[1]{\left[#1\right]}
\newcommand\lo[1]{\mathscr{#1}}
\newcommand\fu[1]{\mathcal{#1}}
\newcommand\bb[1]{\mathbb{#1}}

\newcommand\bs{_\blacksquare}

\newcommand\tv{\lo{V}_\varphi}
\newcommand\tvm{\lo{V}_\varphi^\fu{M}}

% 常用算子
\newcommand{\EE}{\mathbb E}
\newcommand{\PP}{\mathbb P}
\DeclareMathOperator{\dom}{dom}

% 绝对值、范数、内积
\newcommand\abs[1]{\left|#1\right|}
\newcommand\norm[1]{\left\lVert#1\right\rVert}
\newcommand\inp[1]{\left\langle#1\right\rangle}
\newcommand\vct[1]{\mathbf{#1}}

% 彩色交叉引用
\newcommand\bpro[1]{{\small\color{blue}Pro.\ref{#1}}}
\newcommand\btheo[1]{{\small\color{blue}Thm.\ref{#1}}}
\newcommand\plem[1]{{\small\color{mauve}Lem.\ref{#1}}}
\newcommand\gname[1]{{\small\color{gd}{#1}}}

%==== 全局参数 ==================================================================
\setcounter{tocdepth}{2}    % 目录显示到 section
\tolerance=10000            % 断行容忍度



\begin{document}


\pagenumbering{roman}

\tableofcontents

\clearpage
\pagenumbering{arabic}
\setlist[enumerate]{label=\arabic*),topsep=0pt,parsep=0pt,wide=0pt,itemsep=0pt,partopsep=0pt}
\setlist[itemize]{topsep=0pt,parsep=0pt,wide=0pt,itemsep=0pt,partopsep=0pt}
\arraycolsep=1.4pt

\pretocmd{\subsection}{%
  \setlength{\abovedisplayskip}{3pt}%
  \setlength{\belowdisplayskip}{3pt}%
  \setlength{\abovedisplayshortskip}{3pt}%
  \setlength{\belowdisplayshortskip}{3pt}%
}{}{}
\renewcommand{\thefootnote}{\fnsymbol{footnote}}
\small




\section{Lagrangian Mechanics}
\qquad Let us start from defining the {\bf action}:
\[S\e{\bm{q}\f{t}}=\int_0^TL\f{\bm{q},\dot{\bm{q}},t}\,dt,\]
the {\bf Eular-Lagrange equation} is derived from $\delta S=0$, with an additional restriction $\delta{\bm{q}\f{0}}=\delta{\bm{q}\f{T}}=0$:
\[\begin{array}{rcl}
\displaystyle\delta\int_0^TL\,dt&=&\displaystyle\int_0^T\f{\nabla_{\bm{q}}L\cdot\delta\bm{q}+\nabla_{\dot{\bm{q}}}L\cdot\delta\dot{\bm{q}}}\,dt\\
&=&\displaystyle\int_0^T\f{\nabla_{\bm{q}}L-\frac{d}{dt}\nabla_{\dot{\bm{q}}}L}\cdot\delta\bm{q}\,dt=0.
\end{array}\]
Since the choice of $\delta\bm{q}$ is arbitrary, we obtain the Eular-Lagrange equation (we place the highest order derivative at the beginning): 
\begin{equation}
\frac{d}{dt}\bm{\nabla}_{\dot{\bm{q}}}L-\bm{\nabla}_{\bm{q}}L=\bm{0}.
\end{equation}
But what is the {\bf Lagrangian} function $L$?
\subsection{Free Particle}
\qquad First, there are some symmetry about the Lagrangian:
\begin{enumerate}
\item Space translation: $L\f{\bm{q},\dot{\bm{q}},t}=L\f{\dot{\bm{q}}}$.
\item Rotation: $L\f{\dot{\bm{q}}}=L\f{\abs{\dot{\bm{q}}}^2}$.
\end{enumerate}
In this sense:
\[2\frac{d}{dt}L'\f{\abs{\dot{\bm{q}}}^2}\dot{\bm{q}}.\]
Compare to the Newtonian Mechanics:
\[m\ddot{\bm{q}}=0,\]
we make $L'\f{\abs{\dot{\bm{q}}}^2}$ as a constant $m/2$:
\begin{equation}
L_\text{free}=T=\frac{1}{2}m\abs{\dot{\bm{q}}}^2.
\end{equation}
\subsection{Conservative Force}
\qquad In the case of conservative force:
\[m\ddot{\bm{r}}=\bm{F}=-\bm{\nabla}V\f{\bm{q}}.\]
Then we have
\[\frac{d}{dt}\bm{\nabla}_{\dot{\bm{q}}}T-\bm{\nabla}_{\bm{q}}T+\bm{\nabla}V\f{\bm{q}}=\frac{d}{dt}\bm{\nabla}_{\dot{\bm{q}}}\f{T-V}-\bm{\nabla}_{\bm{q}}\f{T-V}=0.\]
So we define
\begin{equation}
L_\text{conserve}=T-V=\frac{1}{2}m\abs{\dot{\bm{q}}}^2-V.
\end{equation}
\subsection{Constraint Force}
\qquad Under the following conditions, we can use the {\bf generalized coordinate}:
\begin{enumerate}
\item There is some holonomic constraint $f\f{\bm{q},t}$.
\item The constraint force satisfies $\bm{F}\cdot\delta\bm{q}=0$.
\end{enumerate}
In Newtonian Mechanics:
\[m\ddot{\bm{q}}=\bm{F}_\text{conserve}+\bm{F}_\text{constraint}.\]
But in the variation of the action:
\[\begin{array}{rcl}
\delta S&=&\displaystyle\int_0^T\f{m\ddot{\bm{q}}-\bm{F}_\text{conserve}-\bm{F}_\text{constraint}}\cdot\delta\bm{q}\,dt\\
&=&\displaystyle\int_0^T\f{m\ddot{\bm{q}}-\bm{F}_\text{conserve}}\cdot\delta\bm{q}\,dt=0
\end{array}\]






\chapter{Introduction to Statistics}
\subfile{../3_computer_science/python/python.tex}









\onecolumn
\begin{thebibliography}{999}

\end{thebibliography}



\end{document}