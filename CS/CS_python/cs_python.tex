%==== 文档类 ====================================================================
\documentclass[../../main.tex]{subfiles}

\begin{document}
\setlist[enumerate]{label=\arabic*),topsep=0pt,parsep=0pt,wide=0pt,itemsep=0pt,partopsep=0pt}
\setlist[itemize]{topsep=0pt,parsep=0pt,wide=0pt,itemsep=0pt,partopsep=0pt}
\arraycolsep=1.4pt








\section{Data Type}
\subsection{Object}
\qquad Python is an {\bf object-oriented} programing language. Everything is an {\bf object} in Python:
\[\text{object}=\left\{\begin{array}{l}
\text{identity},\\
\text{type / class},\\
\text{value / state},\\
\text{methods / behaviors / operations}.\\
\end{array}\right.\]
\begin{lstlisting}[language=Python]
# print the identity, type, and the value for 4
print(id(4),type(4),4)
# type of any type is a type, the type itself is a type
print(type(type(4)))
print(type(type(type(4))))
\end{lstlisting}
\begin{pyout}
140711773227544 <class 'int'> 4
<class 'type'>
<class 'type'>
\end{pyout}
\begin{itemize}
\item {\bf Identity}: it guarantees that {different objects} have {distinct identities} at {any given time}. 
\item {\bf Type}: objects of the same type support the same {operations}, and share the same {properties}.
\end{itemize}


\subsection{Binding and Input}
\qquad In Python, the {\bf assignment} of $a=b$ is like making the name {\itshape a} pointing to the object {\itshape b}.
\begin{lstlisting}[language=Python]
# an example for binding
a,b=4,print
print(type(a),a,type(b),b,
      id(a),id(4),id(b),id(print))
b(a+5,"hello")
\end{lstlisting}
\begin{pyout}
<class 'int'> 4 <class 'builtin_function_or_method'> <built-in function print> 140723891816984 140723891816984 2069908885472 2069908885472
9 hello
\end{pyout}
\qquad The basic input in Python is through the function input( ). The input takes ONE string as prompt, and it reads input as a string.
\begin{lstlisting}[language=Python]
# an example for input function
n=input(f"{a} and hello\n")
print(type(n),n)
\end{lstlisting}
\begin{pyout}
4 and hello
5
<class 'str'> 5
\end{pyout}



\subsection{Numeric}
\qquad The following are numeric types:
\[\text{bool}\subset\text{int}\subset_?\text{float}\subset_?\text{complex}\]
\begin{lstlisting}[language=Python]
# an example for the above data types
print(type(True),True,type(1),1,
      type(1.0),1.0,type(1+0j),1+0j)
\end{lstlisting}
\begin{pyout}
<class 'bool'> True <class 'int'> 1 <class 'float'> 1.0 <class 'complex'> (1+0j)
\end{pyout}
\begin{lstlisting}[language=Python]
# subset example
if True==1==1.0==1+0j:
    print("Yes")
else:
    print("No")
\end{lstlisting}
\begin{pyout}
Yes
\end{pyout}
\qquad We can use bool( ), int( ), float( ), and complex( ) to convert a string to the corresponding data type from input( ); 
\begin{lstlisting}[language=Python]
# input string to number
n=input("type in an integer\n")
print(type(n),n,type(int(n)),int(n))
\end{lstlisting}
\begin{pyout}
type in an integer
17
<class 'str'> 17 <class 'int'> 17
\end{pyout}
identically map from a subset to a larger set, or canonically map from the supset to the restricted set:
\begin{lstlisting}[language=Python]
# identical map and canonical map
print(int(False),float(5),int(3.7))
\end{lstlisting}
\begin{pyout}
0 5.0 3
\end{pyout}
\subsection{More on Bool}
\begin{lstlisting}[language=Python]
# logic and bool
print(type(1==0))
print(type(""),bool(""))
if not "":
    print("statement or bool value defined can be used in logic")
\end{lstlisting}
\begin{pyout}
<class 'bool'>
<class 'str'> False
statement or bool value defined can be used in logic
\end{pyout}
\qquad For statements and numbers, there is a {\bf special method} bool:
\begin{lstlisting}[language=Python]
# special method __bool__()
print(type((5==3).__bool__()),(5==3).__bool__(),
      id((5==3).__bool__()),id(False))
\end{lstlisting}
\begin{pyout}
<class 'bool'> False 140723890821168 140723890821168
\end{pyout}
\begin{lstlisting}[language=Python]
# special method __bool__() for numbers
print(type((0+3.5j).__bool__()),(0+3.5j).__bool__(),
      id((0+3.5j).__bool__()),id(True))
\end{lstlisting}
\begin{pyout}
<class 'bool'> True 140723890821136 140723890821136
\end{pyout}
For some data like string, list etc., the special method len is defined.
\begin{lstlisting}[language=Python]
# sepcial method __len__() for strings
print(type("abcd".__len__()),"abcd".__len__(),
      id("abcd".__len__()),id(4))
\end{lstlisting}
\begin{pyout}
<class 'int'> 4 140723891816984 140723891816984
\end{pyout}
\qquad The bool( ) has a {\bf protocol}:
\begin{enumerate}
\item If special method bool is defined, then return.
\item Else if the special method is defined, then return True if len is not 0 and vise versa.
\item Else return True.
\end{enumerate}
In general, bool( ) is used for logical determination.


\subsection{More on Float}
\qquad For float, there are some useful {\bf regular methods}:
\begin{lstlisting}[language=Python]
# regular method is_integer() for floats
print(type((1.3).is_integer()),(1.3).is_integer())
\end{lstlisting}
\begin{pyout}
<class 'bool'> False
\end{pyout}
\begin{lstlisting}[language=Python]
# regular method as_integer_ratio() for floats
print(type((0.5).as_integer_ratio()),(0.5).as_integer_ratio())
\end{lstlisting}
\begin{pyout}
<class 'tuple'> (1, 2)
\end{pyout}



























































\end{document}