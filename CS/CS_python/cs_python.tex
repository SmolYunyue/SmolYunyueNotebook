%==== 文档类 ====================================================================
\documentclass[../../main.tex]{subfiles}

\begin{document}
\setlist[enumerate]{label=\arabic*),topsep=0pt,parsep=0pt,wide=0pt,itemsep=0pt,partopsep=0pt}
\setlist[itemize]{topsep=0pt,parsep=0pt,wide=0pt,itemsep=0pt,partopsep=0pt}
\arraycolsep=1.4pt








\section{Data Type}
\subsection{Object}
\qquad Python is an {\bf object-oriented} programing language. Everything is an {\bf object} in Python:
\[\text{object}=\left\{\begin{array}{l}
\text{identity},\\
\text{type / class},\\
\text{value / state},\\
\text{methods / behaviors / operations}.\\
\end{array}\right.\]
\begin{lstlisting}[language=Python]
# print the identity, type, and the value for 4
print(id(4),type(4),4)
# type of any type is a type, the type itself is a type
print(type(type(4)))
print(type(type(type(4))))
\end{lstlisting}
\begin{pyout}
140711773227544 <class 'int'> 4
<class 'type'>
<class 'type'>
\end{pyout}
\begin{itemize}
\item {\bf Identity}: it guarantees that {different objects} have {distinct identities} at {any given time}. 
\item {\bf Type}: objects of the same type support the same {operations}, and share the same {properties}.
\end{itemize}


\subsection{Binding and Input}
\qquad In Python, the {\bf assignment} of $a=b$ is like making the name {\itshape a} pointing to the object {\itshape b}.
\begin{lstlisting}[language=Python]
# an example for binding
a,b=4,print
print(type(a),a,type(b),b)
b(a+5,"hello")
\end{lstlisting}
\begin{pyout}
<class 'int'> 4 <class 'builtin_function_or_method'> <built-in function print>
9 hello
\end{pyout}
\qquad The basic input in Python is through the function input( ). The input takes ONE string as prompt, and it reads input as a string.
\begin{lstlisting}[language=Python]
# an example for input function
n=input(f"{a} and hello\n")
print(type(n),n)
\end{lstlisting}
\begin{pyout}
4 and hello
5
<class 'str'> 5
\end{pyout}



\subsection{Numeric}
\qquad The following are numeric types:
\[\text{bool}\subset\text{int}\subset\text{float}\subset\text{complex}\]
\begin{lstlisting}[language=Python]
# an example for the above data types
print(type(True),True,type(1),1,
      type(1.0),1.0,type(1+0j),1+0j)
\end{lstlisting}
\begin{pyout}
<class 'bool'> True <class 'int'> 1 <class 'float'> 1.0 <class 'complex'> (1+0j)
\end{pyout}
\begin{lstlisting}[language=Python]
# subset example
if True==1 and 1==1.0 and 1.0==1+0j:
    print("Yes")
else:
    print("No")
\end{lstlisting}
\begin{pyout}
Yes
\end{pyout}






























































\end{document}