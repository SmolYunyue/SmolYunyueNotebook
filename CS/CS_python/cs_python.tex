%==== 文档类 ====================================================================
\documentclass[../../main.tex]{subfiles}

\begin{document}
\setlist[enumerate]{label=\arabic*),topsep=0pt,parsep=0pt,wide=0pt,itemsep=0pt,partopsep=0pt}
\setlist[itemize]{topsep=0pt,parsep=0pt,wide=0pt,itemsep=0pt,partopsep=0pt}
\arraycolsep=1.4pt








\section{Data Type}
\subsection{Object}
\qquad Python is an {\bf object-oriented} programing language. Everything is an {\bf object} in Python:
\[\text{object}=\left\{\begin{array}{l}
\text{identity},\\
\text{type / class},\\
\text{value / state},\\
\text{methods / behaviors / operations}.\\
\end{array}\right.\]
\begin{lstlisting}[language=Python]
# print the identity, type, and the value for 4
print(id(4),type(4),4)
# type of any type is a type, the type itself is a type
print(type(type(4)))
print(type(type(type(4))))
\end{lstlisting}
\begin{pyout}
140711773227544 <class 'int'> 4
<class 'type'>
<class 'type'>
\end{pyout}
\begin{itemize}
\item {\bf Identity}: it guarantees that {different objects} have {distinct identities} at {any given time}. 
\item {\bf Type}: objects of the same type support the same {operations}, and share the same {properties}.
\end{itemize}


\subsection{String}











\subsection{Numeric}
\qquad The following are numeric types:
\[\text{bool}\subset\text{int}\subset\text{float}\subset\text{complex}\]
\begin{lstlisting}[language=Python]
# an example for the above data types
print(True,1,1.0,1+0j)
\end{lstlisting}
\begin{pyout}
True 1 1.0 (1+0j)
\end{pyout}
\begin{lem}
If $f\in\s{\text{bool},\text{int},\text{float}}$
\end{lem}





\end{document}